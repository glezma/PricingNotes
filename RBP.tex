
\chapter{Risk Based Pricing}
\section{ Preliminary definitions} 

\renewcommand{\arraystretch}{1.5} % <-- optional a
\begin{center} % <-- optional b
\[ % <-- step a
\begin{array}{|l|c|l|} \hline % <-- steps b, c; optional c, d
\mbox{Element} & \mbox{Notation}\\ \hline
\mbox{Client Interest Rate}  & r \\
\mbox{Cost of Funds Rate,Fund Transfer Pricing FTP, TT   }  & r_c \\
\mbox{Discount Rate }  & r_d\\
\mbox{Contractual Maturity }  & T \\
\hline
\end{array} % <-- step e
\] % <-- step e
\end{center}

\subsection{Default and Prepayment probabilities:}
$ \forall t \in \{1,2,...T\}$
\[ 
\begin{array}{rr} 
   p_p(t)  :&  \mbox{
   Probability that the loan will prepay at time t given that it has survived to that point } \\
   p_d(t)  : &  \mbox{
   Probability that the loan will default at time t given that it has survived to that point
   }
\end{array} \noindent
\] 

 \subsection{Survival function:}
$S(t)$ :   Probability that a loan survives until period t 
\begin{align}
S(t) & = \Pi_{s=1}^t (1-p_d(s) - p_p(s)) \\
 & = (1-p_d(1) - p_p(1))\times(1-p_d(2) - p_p(2))\times...\times(1-p_d(t) - p_p(t)) \nonumber
\end{align}

\subsection{Balance function: }
The Current Balance function $\bar{B}(t)$ is the remaining balance left at time $t-1$ for a loan with principal $B=\bar{B}(1)$ and in absence of any prepayment or default risk. For non conventional loan payments this function might not have a closed form solution. 


\paragraph{Constant installments:} The remaining balance at time $t$ for a loan with principal (Balance at t=0) B is given by 
\begin{align}
\bar{ B}(t)&=B\frac{(1+r)^T-(1+r)^{t-1}}{(1+r)^T-1}
\end{align}
\begin{align}
\bar{ I}(t)&=r\times \bar{ B}(t)
\end{align}
The acute reader will notice that the definition of $\bar{B}(t)$, in terms of the remaining balance left at time $t-1$ , was given so that we can state such a simple equation for $\bar{I}(t)$.


\paragraph{Constant amortization: } The remaining balance at time $t$ for a loan with principal (Balance at t=0) B is given by:

\begin{align}
\bar{B}(t)&=B \times (1-\frac{t-1}{T})
\end{align}



%Page on setup equations
\section{Financial Math operators}
\subsection{Constant Installments}
We can define a $c_f$ function to compute constant installments by defining the following:
\begin{align}
    \bar{B}(1) &= \frac{c}{(1+r)}+\frac{c}{(1+r)^2}+\frac{c}{(1+r)^3}+...++\frac{c}{(1+r)^T} \nonumber\\
\end{align}
Setting $\delta=\frac{1}{1+r}$
\begin{align}
    \bar{B}(1) &==c\delta[1+\delta+\delta^2+\delta^3+\delta^4+...+\delta^{T-1}] \nonumber \\
    &=c\delta\left[\frac{1}{1-\delta}-\delta^T\frac{1}{1-\delta}\right] =c\delta\left(\frac{1-\delta^T}{1-\delta}\right)
\end{align}
\begin{align}
    c=\bar{B}(1)\left(\frac{1-\delta}{\delta}\right)\frac{1}{1-\delta^T}=B(1)\left[r\frac{(1+r)^T}{(1+r)^T-1}\right]
\end{align}

\begin{align}
    \boxed{c_f(r,T):=\frac{r(1+r)^T}{(1+r)^T-1}} \implies c=\bar{B}(1)c_f(r,T)
\end{align}
Balance factor:
\begin{align}
    \bar{B}(t)=\bar{B}(1)\left[ \frac{(1+r)^T-(1+r)^{t-1}}{(1+r)^T-1} \right]
\end{align}
\begin{align}
    \boxed{B_f(t,r,T)=\left[ \frac{(1+r)^T-(1+r)^{t-1}}{(1+r)^T-1} \right]} \implies \bar{B}(t)=\bar{B}(1)B_f(t,r,T)
\end{align}
\begin{align}
    \bar{I}(t) = r\bar{B}(1)B_f(t,r,T)
\end{align}
Amortization factor:
\begin{align}
    A_f(t,r,T) = c_f(r,T)-rB_f(t,r,T)
\end{align}
\begin{align}
    \boxed{A_f(t,r,T)=\frac{r(1+r)^{t-1}}{(1+r)^T-1}} \label{eq:afactor}
\end{align}


\begin{theorem}{1.1 (Telescopic Amortizations)}{} \label{teo:1}
If A is the function defined in (\ref{eq:afactor}) then:
\[
\prod^{t-1}_{s=1}(1-A(1,r,T-s+1))=1-\sum^{t-1}_{s=1}A(s,r,T)
\]
\end{theorem}

\begin{proof}{}{} Lets define
\[
E_1=\prod^{t-1}_{s=1}(1-A(1,r,T-s+1)), \\
E_2 =1-\sum^{t-1}_{s=1}A(s,r,T)
\]
and
\[
\delta =1/(1+r)
\]
Working on $E_1$ and setting $\xi=1+r$:
\begin{align}
1-A(1,r,T-s+1) = \frac{(1+r)^{T-s+1}-1-r}{(1+r)^{T-s+1}-1}=\frac{\xi^{T-s+1}-\xi}{\xi^{T-s+1}-1} \nonumber
\end{align}

\begin{align}
\implies E_1 =& \frac{(\xi^T-\xi)}{(\xi^T-1)}\times \frac{(\xi^{T-1}-\xi)}{(\xi^{T-1}-1)} \times \frac{(\xi^{T-2}-\xi)}{(\xi^{T-2}-1)} \times ... \times \frac{(\xi^{T-t+1}-\xi)}{(\xi^{T-t+1}-1)} \nonumber\\
=&\xi^t \frac{(\xi^{T-1}-1)}{(\xi^T-1)}\times \frac{(\xi^{T-2}-1)}{(\xi^{T-1}-1)} \times \frac{(\xi^{T-3}-1)}{(\xi^{T-2}-1)} \times ... \times \frac{(\xi^{T-t}-1)}{(\xi^{T-t+1}-1)} \nonumber\\
=&\frac{\xi^T-\xi^t}{\xi^T-1} \label{eq:e1}
\end{align}

Working on $E_2$:
\begin{align}
A(s,r,T) = \frac{r(1+r)^{s-1}}{(1+r)^{T}-1}=(\xi-1)\frac{\xi^{s-1}}{\xi^{T}-1} \nonumber
\end{align}

\begin{align}
\implies E_2 =& 1-\frac{(\xi-1)(1+\xi+\xi^2+\xi^3+...+\xi^{t-1})}{\xi^T-1}\nonumber\\
=&\frac{\xi^T-\xi^t}{\xi^T-1} 
\end{align}
$\therefore E1=E2$
\end{proof}

%  Page on setup examples
\section{Alternative setups for Incremental Profit computation}



\subsection{Standard model \label{can}} 
In this model prepayments/full prepayment, default probability are expressed as conditional probabilities. This probabilities are conditioned on the running active balance i.e. the balance that has not been prepaid or defaulted upon.
\begin{figure}[H]
  \centering
      \includegraphics[width=.3\textwidth]{Graph2.pdf} 
 \caption{Computation Graph}
 \label{fig:graph2}
\end{figure}

\begin{center} % <-- optional b
%\[ % <-- step a
\begin{array}{|l|c|l|} \hline % <-- steps b, c; optional c, d
\mbox{Variable} &\mbox{Notation} & \mbox{Calculation}\\ \hline
\mbox{Balance in presence of risk }  & B(t)  & B(t)\\
\mbox{Default  }  & D(t) & p_d(t) B(t)\\
\mbox{Full Prepayment}  & C(t) & p_c(t) B(t)\\
\mbox{Prepayment  }  & P(t) & p_p(t)B(t)\\
\mbox{Amortization}  & A(t) &(1-p_d(t)-p_c(t)-p_p(t)) B(t)A_f(1,r,T-t+1))\\
\mbox{Interest }  & I(t) & (1-p_d(t)-p_c(t)-p_p(t))B(t)r\\
\mbox{Principal   }  &  B(1) & B\\
\hline
\end{array} % <-- step e
%\] % <-- step e
\captionof{table}{Computation for Model 1: Standard Model} 
\end{center}

Notice that we defined $B(t)$ as the Loan Balance subject to risk (Conductual affected Loan Balance), as opposed to $\bar{B}(t)$, which is the Risk Free Loan Balance (Contractual Balance).  Given this definitions we can compute the recursive form for the balance function $B(t)$
\begin{align}
B(t+1) =& B(t)[1-p_d(t)-p_c(t)-p_p(t)-(1-p_d(t)-p_c(t)-p_p(t))A(1,r,T-t+1) ] \nonumber\\
     =&
    B(t)(1-p_d(t)-p_c(t)-p_p(t))(1-A(1,r,T-t+1)) \label{eq:bbar1}\
\end{align}
Notice that (\ref{eq:bbar1}) is a first order equation in difference which can be easily solved as.
\begin{align}
    B(t) =\prod^{t-1}_{s=1} (1-A(1,r,T-s+1))(1-p_d(s)-p_c(s)-p_p(s))B
\end{align}
Using theorem (\ref{teo:1}) we can state the conductual balance $B(t)$ as a function of the contractual balance.
\begin{align}
    B(t) &=\prod^{t-1}_{s=1} (1-p_d(s)-p_c(s)-p_p(s))\bar{B}(t) \nonumber\\
    &=S(t)\bar{B}(t) \nonumber
\end{align}
\begin{align}
    \boxed{B(t)=S(t-1)\bar{B}(t) } \label{eq:model1}
\end{align}
Equation \ref{eq:model1} states a very simple relation between the theoretical/contractual balance $\bar{B}(t)$ and the behavioral balance $B(t)$. We have derived this equation from a first principles approach through the Theorem of Telescopic Amortization (\ref{teo:1}). This equation represent a very powerful shortcut not only for using intuition, since the behavioral balance can be thought of as the contractual balance adjusted by the survival probability, but also for implementing the model programatically using vectorization instead of recursive loops over the different points in the payment schedule, the last alternative can be very hard to maintain and compute, not to mention its proneness to error.

\subsection{Prepayment dependent on initial balance}
This model is a variation of the previous one in which the prepayment probabiliy is a proportion of the initial balance/principal $B(1)$
\begin{figure}[H]
  \centering
      \includegraphics[width=.3\textwidth]{Graph1.pdf} 
 \caption{Computation Graphs: Model 2}
 \label{fig:Test}
\end{figure}

\begin{center} % <-- optional b
% \[ % <-- step a
\begin{array}{|l|c|l|} \hline % <-- steps b, c; optional c, d
\mbox{Variable} &\mbox{Notation} & \mbox{Calculation}\\ \hline
\mbox{Balance in presence of risk }  & B(t)  & B(t)\\
\mbox{Default  }  & D(t) & p_d(t) B(t)\\
\mbox{Amortization}  & A(t) &(1-p_d(t)) B(t)A(1,r,T-t+1))\\
\mbox{Interest }  &  I(t) & (1-p_d(t))B(t)r\\
\mbox{Full Prepayment}  & C(1) & p_c(t) B(t)[1-p_d(t)-(1-p_d(t))A(1,r,T-t+1)]\\
\mbox{Principal   }  &  B(1) & B\\
\mbox{Prepayment  }  & P(t) & p_p(t)B\\
\hline
\end{array} % <-- step e
% \] % <-- step e
\captionof{table}{Computation for Model 2: Prepayment Dependent on Initial Balance} 
\end{center}

  Given this definitions we can compute the recursive form for the balance function $B(t)$
\begin{align}
\scriptstyle
     \bar{B}(t+1) =&\scriptstyle \bar{B}(t)[1-p_d(t)-p_c(t)(1-p_d(t)-(1-p_d(t))A(1,r,T-t+1))-(1-p_d(t))A(1,r,T-t+1) ]-p_p(t) \bar{B}(1) \nonumber\\
    =&\scriptstyle \bar{B}(t)[1-p_d(t)-p_c(t)+p_d(t)p_c(t)+p_c(t)(1-p_d(t)A(1,r,T-t+1)-(1-p_d(t))A(1,r,T-t+1)]
    -p_p(t) \bar{B}(1) \nonumber\\
    =&\scriptstyle
    \bar{B}(t)[ (1-p_d(t))(1-p_c(t))-(1-p_d(t))(1-p_c(t))A(1,r,T-t+1)]
    -p_p(t) \bar{B}(1) \nonumber\\
     =&\scriptstyle
    \bar{B}(t)(1-p_d(t))(1-p_c(t))(1-A(1,r,T-t+1))
    -p_p(t) \bar{B}(1) \label{eq:bbar}\
\end{align}
Notice that (\ref{eq:bbar}) is a first order equation in difference of type $x(t+1) = \alpha(t)x(t) + \beta(t)$  where $\alpha(t):=(1-p_d(t))(1-p_c(t))(1-A(1,r,T-t+1))$  , $\beta(t):= -p_p(t) \bar{B}(1)$, $x(t):= \bar{B}(t)$ and the initial condition given by the lended principal $x(1) = \bar{B}(1) = B$. This equation can be easily solved as: 
\begin{align}
    x(t+1) = x_1 \prod_{s=1}^t \alpha(s) + \sum_{k=1}^{t-2} \left[\beta(k) \prod_{s=k+1}^{t-1}\alpha(s)\right]+\beta(t-1)
\end{align}
Plugging back the definitions of $\alpha(t)$, $\beta(t)$ and ,$x(t)$ we get:
\begin{align}
    B(t+1) =&\prod^{t}_{s=1} (1-p_d(s))(1-p_p(s))(1-A(1,r,T-s+1))B \nonumber\\
            &-\sum_{k=1}^{t-1}\left[p_p(t) B \prod_{s=k+1}^{t-1}(1-p_d(t))(1-p_c(t))(1-A(1,r,T-t+1))\right]-p_p(t-1) B
\end{align}
Using the theorem of Telescopic Amortizations (\ref{teo:1}), the definitions for survival curve and lagging t one period we can state the conductual balance $B(t)$ as a function of the contractual balance $\bar{B}(t)$.
\begin{align}
    B(t) &=\prod^{t-1}_{s=1} (1-p_d(s))(1-p_p(s))\bar{B}(t)-\sum_{k=1}^{t-2} \left[ p_p(k)B \frac{S(t-2) }{S(k) }  \frac{ \bar{B}(t-1)}{ \bar{B}(k+1)}\right] \nonumber\\
    &=S(t-1)\bar{B}(t)-\sum_{k=1}^{t-2} \left[ p_p(k)B \frac{S(t-2) }{S(k) }  \frac{ \bar{B}(t-1)}{ \bar{B}(k+1)}\right] \nonumber 
\end{align}

\begin{align}
    \boxed{B(t)=S(t-1)\bar{B}(t)-\sum_{k=1}^{t-2} \left[ p_p(k)B \frac{S(t-2) }{S(k) }  \frac{ \bar{B}(t-1)}{ \bar{B}(k+1)}\right]  } \label{eq:model2}
\end{align}
Equation (\ref{eq:model2}) is analogous to equation (\ref{eq:model1}) with an aditional term that represents a weighted sum of the last prepayment amounts where the last prepaymet has a weight of one. Equivalently, instead of excluding the prepayment amount through the survival factor $S(t-1)$ we are substracting these amounts outside the $S(t-1)\bar{B}(t)$ factor.
\section{ Terms included in the incremental profit (CLV)}
\subsection{ Interest on loans: }
\begin{align}
LI(t) = S(t)\bar{ B}(t)r \label{eq:li}
\end{align}
Equation (\ref{eq:li}) will be proved in section (\ref{can}). For now lets just use our intuition and state that each dollar has an unconditional probability to survive up to time $t$ of $S(t)$ 
\subsection{  Cost of Funds: }
\begin{align}
COF(t) = S_c(t)\bar{ B}_c(t)r_c
\end{align}
Where: 
\begin{align}
S_c(t)= \Pi_{s=0}^t [1- p_p(s)-(1-LGD(s))p_d(s) ]
\end{align}
The last formula can be viewed as the complement for the probability of death for 1 dollar. If a prepayment event 

\subsection{ Equity Benefit (Capital Rebate): }
\begin{align}
EB(t) = \alpha S(t)\bar{ B}(t) r_c
\end{align}

\subsection{ Fees Additional Source of revenue: }
\begin{align}
F(t) = f S(t)
\end{align}

\subsection{ Servicing Costs: }
\begin{align}
SC(t) =  \sigma S(t)
\end{align}

\subsection{ Loss from Default: }
\begin{align}
EL(t) =  p_d(t)LGD(t)S(t)\bar{ B}(t) 
\end{align}
\subsection{Recovery costs}
\begin{align}
C(t) = c\times p_d(t) S(t)
\end{align}

\subsection{ Equity Capital Charge: }
\begin{align}
 EC(t) =  \alpha S(t)\bar{ B}(t) r_e
\end{align}


\section{ Incremental Profit Definition (CLV): }
The net present value is given by:

\begin{align}
NPV(x(t),r,T)=\sum_{t=1}^T \frac{x(t)}{(1+r)^t}
\end{align}

\renewcommand{\arraystretch}{1.5} % <-- optional a
\begin{center} % <-- optional b
\[ % <-- step a
\begin{array}{|l|c|l|} \hline % <-- steps b, c; optional c, d
\mbox{Element} & \mbox{Notation} & \mbox{Calculation}\\ \hline
\mbox{Lending Interest }  & LI & NVP(LI(t),r_d,T)\\
\mbox{Cost of Funds   }  & COF & NVP(COF(t),r_d,T)\\
\mbox{Equity benefit }  & EB & NVP(EB(t),r_d,T)\\
\mbox{Fees }  & LI & NVP(F(t),r_d,T)\\
\mbox{Ancillary profit }  & A & -\\
\mbox{Origination cost}  & OC & -\\
\mbox{Commision  }  & COM & -\\
\mbox{Servicing Costs}  & SC & NVP(SC(t),r_d,T)\\
\mbox{Expected Loss }  & EL & NVP(EL(t),r_d,T)\\
\mbox{Collection costs }  & C & NVP(C(t),r_d,T)\\
\mbox{Equity charge}  & EC & NVP(EC(t),r_d,T)\\

\hline
\end{array} % <-- step e
\] % <-- step e
\end{center}

\renewcommand{\arraystretch}{1.5} % <-- optional a
\begin{center} % <-- optional b
\[ % <-- step a
\begin{array}{|l|c|l|} \hline % <-- steps b, c; optional c, d
\mbox{Element} & \mbox{Notation} & \mbox{Calculation}\\ \hline
\mbox{Net Interest Income }  & NII & LI-COF+EB\\
\mbox{Total Income  }  & TI & NII+A+F\\
\mbox{Net Income before tax}  & NIBT & TI-OC-COM-SC-LD-C\\
\mbox{Net Income after tax }  & NIAT&(1-\tau)\times NIBT \\
\mbox{Incremental profit  }  & IP & NIAT-EC\\

\hline
\end{array} % <-- step e
\] % <-- step e
\end{center}

\subsection{ Incremental Profit Function: }
Define the incremental profit function as:
\begin{align}
\pi(p)=IP(p) \label{eq:IP}
\end{align}
