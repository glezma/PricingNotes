
\documentclass[12pt]{article}
 
\usepackage[margin=1in]{geometry} 
\usepackage{amsmath,amsthm,amssymb}
 
\newcommand{\N}{\mathbb{N}}
\newcommand{\Z}{\mathbb{Z}}
 
\newenvironment{theorem}[2][Theorem]{\begin{trivlist}
\item[\hskip \labelsep {\bfseries #1}\hskip \labelsep {\bfseries #2.}]}{\end{trivlist}}
\newenvironment{lemma}[2][Lemma]{\begin{trivlist}
\item[\hskip \labelsep {\bfseries #1}\hskip \labelsep {\bfseries #2.}]}{\end{trivlist}}
\newenvironment{exercise}[2][Exercise]{\begin{trivlist}
\item[\hskip \labelsep {\bfseries #1}\hskip \labelsep {\bfseries #2.}]}{\end{trivlist}}
\newenvironment{problem}[2][Problem]{\begin{trivlist}
\item[\hskip \labelsep {\bfseries #1}\hskip \labelsep {\bfseries #2.}]}{\end{trivlist}}
\newenvironment{question}[2][Question]{\begin{trivlist}
\item[\hskip \labelsep {\bfseries #1}\hskip \labelsep {\bfseries #2.}]}{\end{trivlist}}
\newenvironment{corollary}[2][Corollary]{\begin{trivlist}
\item[\hskip \labelsep {\bfseries #1}\hskip \labelsep {\bfseries #2.}]}{\end{trivlist}}

\newenvironment{solution}{\begin{proof}[Solution]}{\end{proof}}
 
\begin{document}
 
% --------------------------------------------------------------
%                         Start here
% --------------------------------------------------------------
 
\title{Pricing Notes}%replace X with the appropriate number
\author{GL\\ %replace with your name
} %if necessary, replace with your course title
 
\maketitle
 
% \begin{theorem}{x.yz} %You can use theorem, exercise, problem, or question here.  Modify x.yz to be whatever number you are proving
% Delete this text and write theorem statement here.
% \end{theorem}
 
% \begin{proof} %You can also use solution in place of proof.
% Blah, blah, blah.  Here is an example of the \texttt{align} environment:
%Note 1: The * tells LaTeX not to number the lines.  If you remove the *, be sure to remove it below, too.
%Note 2: Inside the align environment, you do not want to use $-signs.  The reason for this is that this is already a math environment. This is why we have to include \text{} around any text inside the align environment.


\subsection{ Preliminary definitions} 
\subsubsection{Default and Prepayment probabilities:}
$ \forall t \in \{1,2,...T\}$
\[ % <-- step a
\begin{array}{rr} % <-- steps b and c
   p_p(t)  :&  \mbox{
   Probability that the loan will prepay at time t given that it has survived to that point } \\
   p_p(t)  : &  \mbox{
   Probability that the loan will default at time t given that it has survived to that point
   }
\end{array} \noindent% <-- step e
\] % <-- step e

 \subsubsection{Survival function:}
$S(t)$ :   Probability that a loan survives until period t 
\begin{align}
S(t) & = \Pi_{s=1}^t (1-p_d(s) - p_p(s)) \\
 & = (1-p_d(1) - p_p(1))\times(1-p_d(2) - p_p(2))\times...\times(1-p_d(t) - p_p(t))
\end{align}

\subsubsection{Balance function: }
The Current Balance function $B(t)$ is the remaining balance left at time $t-1$ for a loan with principal B=B(1). For non conventional loan payments this function might not have a closed form solution. 
-  ** Constant installments: ** The remaining balance at time $t$ for a loan with pricincipal (Balance at t=0) B is given by 
\begin{align}
B(t)&=B\frac{(1+r)^T-(1+r)^{t-1}}{(1+r)^T-1}
\end{align}
\begin{align}
I(t)&=r\times B(t)
\end{align}
The acute reader will notice that the definition of $B(t)$, in terms of the remaining balance left at time $t-1$ , was given so that we can state such a simple equation for $I(t)$.


\subsubsection{Constant amortization: } The remaining balance at time $t$ for a loan with pricincipal (Balance at t=0) B is given by:

\begin{align}
B(t)&=B \times (1-\frac{t-1}{T})
\end{align}


### 2) Terms included in the incremental profit
#### ** Interest on loans: **
\begin{align}
LI(t) = S(t)B(t)r
\end{align}
#### ** Cost of Funds: **
\begin{align}
COF(t) = S_c(t)B_c(t)r_c
\end{align}
Where: 
\begin{align}
S_c(t)= \Pi_{s=0}^t [1- p_p(s)-(1-LGD(s))p_d(s) ]
\end{align}

#### ** Equity Benefit: **
\begin{align}
EB(t) = \alpha S(t)B(t) r_c
\end{align}

#### ** Fees Additional Source of revenue: **
\begin{align}
F(t) = f S(t)
\end{align}

#### ** Servicing Costs: **
\begin{align}
SC(t) =  \sigma S(t)
\end{align}

#### ** Loss from Default: **
\begin{align}
EL(t) =  p_d(t)LGD(t)S(t)B(t) 
\end{align}
\begin{align}
C(t) = c\times p_d(t) S(t)
\end{align}

#### ** Equity Capital Charge: **
\begin{align}
 EC(t) =  \alpha S(t)B(t) r_e
\end{align}


## ** 3) Incremental Profit Definition: **
The net present value is given by:

\begin{align}
NPV(x(t),r,T)=\sum_{t=1}^T \frac{x(t)}{(1+r)^t}
\end{align}
| Element                 | Notation             | Calculation  |
|:------------------------ |:------------------|: -----|
| Lending Interest     | $LI$ | $NVP(LI(t),r_d,T)$ |
| Cost of Funds     | $COF$      |   $NVP(COF(t),r_d,T)$ |
| Equity benefit | $EB$      |    $NVP(EB(t),r_d,T)$ |
| Fees | $F$      |    $NVP(F(t),r_d,T)$ |
| Ancillary profit | $A$      |    - |
| Origination cost | $OC$      |    - |
| Commision | $COM$      |    - |
| Servicing Costs | $SC$      |    $NVP(SC(t),r_d,T)$ |
| Expected Loss| $EL$      |    $NVP(EL(t),r_d,T)$ |
| Collection costs| $C$      |    $NVP(C(t),r_d,T)$ |
| Equity charge| $EC$      |    $NVP(EC(t),r_d,T)$ |

# 

| Element                 | Notation             | Calculation  |
|:------------------------ |:------------------|: -----|
| Net Interest Income  | $NII$      |    $LI-COF+EB$ |
| Total Income  | $TI$      |    $NII+A+F$ |
| Net Income before tax | $NIBT$      |    $TI-OC-COM-SC-LD-C$ |
| Net Income after tax | $NIAT$      |    $(1-\tau)\times NIBT$ |
| Incremental profit | $IP$      |    $NIAT-EC$ |





% \begin{align*}

% \end{align*}
% \end{proof}
% \begin{equation}
%     \alpha+\beta
% \end{equation} 
% \begin{theorem}{x.yz}
% Let $n\in \Z$.  Then yada yada.
% \end{theorem}
 
% \begin{proof}
% Blah, blah, blah.  I'm so smart.
% \end{proof}
 
% --------------------------------------------------------------
%     You don't have to mess with anything below this line.
% --------------------------------------------------------------
 
\end{document}