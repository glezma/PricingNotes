
\chapter{Willingness to Pay Modeling (WTP): }
\section{The Price-Response Function}
Each price-response function specifies the demand that the lender would experience at each price, which will depend on:
\begin{enumerate}
    \item Total number of clients interested in the loan
    \item Number of applicant clients
    \item The number of applicant the lender deems creditworthy and quotes a price.
    \item Number of accepted applicants who would achieve a positive surplus from taking the loan from the lender at the offered price.
    \item Number of accepted applicants who \textbf{take-up} the offered loan.
\end{enumerate}
In most lending markets, the final price is not known to the client at the time she applies for the loan so we assume that the number of clients who apply for a loan is not influenced by the price.
\begin{align}
d(p)=D\bar{F}(p)
\end{align}
$d(p)$ is the number of the loans offered by a lender that would be taken up at the price p. $D$ is the number of successful applicants for the loan, and $\overline{F}(p)$ is the take-up rate, which is defined as the fraction of successful applicants who will take up the loan at price $p$
\begin{align}
d(p)=D\bar{F}(p)
\end{align}
\begin{align}
\bar{F}(p)=\int^\infty_p f(w)dw
\end{align}
\section{Segmented vs Join Estimation}
For $n$ segments, the segmented estimation assumes each segments has its own demand function so we need to estimate $2n$ parameters.
\begin{align}
\bar{F_i}(p_i)=\frac{e^{a_i+b_i p_i}}{1+e^{a_i+b_i p_i}}
\end{align}

For $n$ segments, the join estimation assume we can estimate one single price price response function that includes all explanatory variables within it (including price)
\begin{align}
\bar{F}(p_i,a,b,\theta,x_i)=\frac{e^{a+b p_i+\theta^T x_i}}{1+e^{a+b p_i+\theta^T x_i}}
\end{align}